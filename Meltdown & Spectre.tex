% TEMPLATE FOR SEMINARS
%
% This template serves as a basic style for your seminar paper. Please
% make yourself familiar with LaTeX, BibTeX and this template. Several
% parts of the template need to be adjusted, these places are marked
% with FIXME.
%
% Contact your advisor in case you have questions.

\documentclass[a4paper,oneside,openright] {scrreprt}
\usepackage[T1]{fontenc}
\usepackage[latin1]{inputenc}
\usepackage[pdftex]{graphicx,color}
\usepackage{graphicx}
\usepackage{epsfig}
\usepackage{fancyhdr}
\usepackage{times}
\usepackage{url}
\usepackage{amsmath}
\usepackage{amsfonts}
%\usepackage{dtklogos}
%\usepackage{blindtext}
\usepackage{tikz}
%\usepackage{longtable}
%\usepackage{supertabular}
%\usepackage{verbatim}
%\usepackage{color}
%\usepackage{moreverb}       % z.B. \listinginput
%\usepackage{float}

\newcommand{\HRule}{\rule{\linewidth}{0.5mm}}

\title{\Huge Meltdown \& Spectre}
\author{\Large Marvin Schirrmacher}

\makeatletter

%*** Commands you may want to use ***
\newtheorem{theorem}{Theorem}[section]
\newtheorem{lemma}[theorem]{Lemma}
\newtheorem{proposition}[theorem]{Proposition}
\newtheorem{corollary}[theorem]{Corollary}
\newtheorem{definition}[theorem]{Definition}
\newtheorem{algorithm}[theorem]{Algorithm}
\newenvironment{example}{\begin{quote}{\bf Example:}}{\end{quote}}

\begin{document}

%%% Title page for seminar

\begin{titlepage}

\centering

\begin{tikzpicture}[remember picture,overlay]
\node[shift={(-6cm,-24cm)},opacity=1] {\includegraphics[scale=0.6]{./figures/hgi_syssec}};
\node[shift={(6cm,-24cm)},opacity=1] {\includegraphics[scale=1.5]{./figures/Logo_RUB}};
\end{tikzpicture}
\vspace{2cm}

\textsc{\LARGE Master Seminar}\\[0.2cm]
\textsc{\LARGE ``Current Topics in Computer Security''}\\[0.9cm]
\textsc{\Large Chair for Systems Security}\\[0.5cm]
\textsc{\Large Ruhr-University Bochum}\\[1.5cm]

% Title
\HRule \\[0.4cm]
{ \huge \bfseries \@title}\\[0.4cm]

\HRule \\[1.5cm]

% Author
\begin{minipage}{0.4\textwidth}
\begin{flushleft} \large
\emph{Author:}\\{\@author}\\[1cm]
{\large \today}
\end{flushleft}
\end{minipage}

\end{titlepage}


\pagestyle{plain}
\cleardoublepage
\pagenumbering{roman}
\tableofcontents
\clearpage
\pagenumbering{arabic}

%************************************************************%
%* START WRITING HERE OR PROVIDE SECTIONS IN SEPERATE FILES *%
%************************************************************%

%%%
\chapter{Overview}  
\label{ch:intro}
%%%

XXXXXXXXXXXXXXXXXXXX
XXXXXXXXXXXXXXXXXXXX
XXXXXXXXXXXXXXXXXXXX
XXXXXXXXXXXXXXXXXXXX
XXXXXXXXXXXXXXXXXXXX


\section{Meltdown and Spectre in Brief}
\label{ch:intro:motivation}

XXXXXXXXXXXXXXXXXXXX
XXXXXXXXXXXXXXXXXXXX
XXXXXXXXXXXXXXXXXXXX
XXXXXXXXXXXXXXXXXXXX
XXXXXXXXXXXXXXXXXXXX

\section{Discovery and Research}
\label{ch:intro:discoveryAndResearch}

XXXXXXXXXXXXXXXXXXXX
XXXXXXXXXXXXXXXXXXXX
XXXXXXXXXXXXXXXXXXXX
XXXXXXXXXXXXXXXXXXXX
XXXXXXXXXXXXXXXXXXXX

\section{Relevance and Practical Impact}
\label{ch:intro:motivation}

XXXXXXXXXXXXXXXXXXXX
XXXXXXXXXXXXXXXXXXXX
XXXXXXXXXXXXXXXXXXXX
XXXXXXXXXXXXXXXXXXXX
XXXXXXXXXXXXXXXXXXXX
~\cite{Rivest:83:RSA}
XXXXXXXXXXXXXXXXXXXX
XXXXXXXXXXXXXXXXXXXX
XXXXXXXXXXXXXXXXXXXX
XXXXXXXXXXXXXXXXXXXX
XXXXXXXXXXXXXXXXXXXX
~\cite{Garfinkel:03:VMI}
XXXXXXXXXXXXXXXXXXXX
XXXXXXXXXXXXXXXXXXXX
XXXXXXXXXXXXXXXXXXXX
XXXXXXXXXXXXXXXXXXXX
XXXXXXXXXXXXXXXXXXXX


%%%
\chapter{CPU Architecture Basics}
\label{ch:background}
%%%

XXXXXXXXXXXXXXXXXXXX
XXXXXXXXXXXXXXXXXXXX
XXXXXXXXXXXXXXXXXXXX
XXXXXXXXXXXXXXXXXXXX
XXXXXXXXXXXXXXXXXXXX

\section{Architecture versus Microarchitecture}
\label{ch:intro:motivation}

XXXXXXXXXXXXXXXXXXXX

\section{Traps}
\label{ch:intro:motivation}

XXXXXXXXXXXXXXXXXXXX

\section{Branch Predictor}
\label{ch:intro:motivation}

XXXXXXXXXXXXXXXXXXXX

\section{Reorder Buffer}
\label{ch:intro:motivation}

XXXXXXXXXXXXXXXXXXXX

\section{Address Spaces}
\label{ch:intro:motivation}

XXXXXXXXXXXXXXXXXXXX

\section{Out-Of-Order Execution}
\label{ch:intro:motivation}

XXXXXXXXXXXXXXXXXXXX

%%%
\chapter{Meltdown versus Spectre}
\label{ch:contentI}
%%%

XXXXXXXXXXXXXXXXXXXX
XXXXXXXXXXXXXXXXXXXX
XXXXXXXXXXXXXXXXXXXX
XXXXXXXXXXXXXXXXXXXX
XXXXXXXXXXXXXXXXXXXX

\section{Differences in Brief}
\label{ch:intro:motivation}

XXXXXXXXXXXXXXXXXXXX

\section{What Unites Them?}
\label{ch:intro:motivation}

XXXXXXXXXXXXXXXXXXXX

\subsection{Out-Of-Order Execution}
\label{ch:intro:motivation:A}

XXXXXXXXXXXXXXXXXXXX

\subsection{Covert Channels}
\label{ch:intro:motivation:A}

XXXXXXXXXXXXXXXXXXXX

\subsection{Cache Attacks}
\label{ch:intro:motivation:A}

XXXXXXXXXXXXXXXXXXXX

\section{What Divides Them?}
\label{ch:intro:motivation}

XXXXXXXXXXXXXXXXXXXX

\subsection{Meltdown and the Usage of Exceptions}
\label{ch:intro:motivation:A}

XXXXXXXXXXXXXXXXXXXX

\subsection{Spectre and the Usage of Branching}
\label{ch:intro:motivation:A}

XXXXXXXXXXXXXXXXXXXX

\subsection{Vulnerable Processors}
\label{ch:intro:motivation:A}

XXXXXXXXXXXXXXXXXXXX

%%%
\chapter{Source Code Analysis}
\label{ch:sourceCodeAnalysis}
%%%

XXXXXXXXXXXXXXXXXXXX
XXXXXXXXXXXXXXXXXXXX
XXXXXXXXXXXXXXXXXXXX
XXXXXXXXXXXXXXXXXXXX
XXXXXXXXXXXXXXXXXXXX

\section{Meltdown Implementation}
\label{ch:intro:motivation}

XXXXXXXXXXXXXXXXXXXX


\subsection{Transient Assembler Instructions}
\label{ch:intro:motivation:A}

XXXXXXXXXXXXXXXXXXXX

\subsection{Exception Suppression}
\label{ch:intro:motivation:A}

XXXXXXXXXXXXXXXXXXXX

\subsection{Detecting Cached Bytes}
\label{ch:intro:motivation:A}

XXXXXXXXXXXXXXXXXXXX

\subsection{Retrieving Secret Bytes}
\label{ch:intro:motivation:A}

XXXXXXXXXXXXXXXXXXXX

\section{Spectre Implementation}
\label{ch:intro:motivation}

XXXXXXXXXXXXXXXXXXXX

\subsection{Detecting Cached Bytes}
\label{ch:intro:motivation:A}

XXXXXXXXXXXXXXXXXXXX

\subsection{Mistraining the Branch Predictor}
\label{ch:intro:motivation:A}

XXXXXXXXXXXXXXXXXXXX

\subsection{Retrieving Secret Bytes}
\label{ch:intro:motivation:A}

XXXXXXXXXXXXXXXXXXXX

%%%
\chapter{Countermeasures}
\label{ch:countermeasures}
%%%

XXXXXXXXXXXXXXXXXXXX
XXXXXXXXXXXXXXXXXXXX
XXXXXXXXXXXXXXXXXXXX
XXXXXXXXXXXXXXXXXXXX
XXXXXXXXXXXXXXXXXXXX

\section{Meltdown}
\label{ch:intro:motivation}

\section{Spectre}
\label{ch:intro:motivation}


%%%
\chapter{Conclusion}
\label{ch:conclusion}
%%%

XXXXXXXXXXXXXXXXXXXX
XXXXXXXXXXXXXXXXXXXX
XXXXXXXXXXXXXXXXXXXX
XXXXXXXXXXXXXXXXXXXX
XXXXXXXXXXXXXXXXXXXX

\bibliography{literature}
\bibliographystyle{alpha}

\end{document}
